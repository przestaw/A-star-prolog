% !TeX spellcheck = pl_PL
% !TEX encoding = UTF-8 Unicode

\documentclass[11pt]{article} % use larger type; default would be 10pt

\usepackage[utf8]{inputenc} % set input encoding (not needed with XeLaTeX)
\usepackage[T1]{fontenc} %font encoding ? -- need to clarify that
%\usepackage[polish]{babel}

%%% PAGE DIMENSIONS
\usepackage{geometry} % to change the page dimensions
\geometry{a4paper} % or letterpaper (US) or a5paper or....
\geometry{margin=1in} % for example, change the margins to 2 inches all round

\usepackage{graphicx} % support the \includegraphics command and options
\usepackage[parfill]{parskip} % Activate to begin paragraphs with an empty line rather than an indent

%%% PACKAGES
\usepackage{booktabs} % for much better looking tables
\usepackage{array} % for better arrays (eg matrices) in maths
\usepackage{paralist} % very flexible & customisable lists (eg. enumerate/itemize, etc.)
\usepackage{verbatim} % adds environment for commenting out blocks of text & for better verbatim
\usepackage{subfig} % make it possible to include more than one captioned figure/table in a single float
% These packages are all incorporated in the memoir class to one degree or another...

%%% HEADERS & FOOTERS
\usepackage{fancyhdr} % This should be set AFTER setting up the page geometry
\pagestyle{fancy} % options: empty , plain , fancy
\renewcommand{\headrulewidth}{0pt} % customise the layout...
\lhead{}\chead{}\rhead{}
\lfoot{}\cfoot{\thepage}\rfoot{}
%%% END Article customizations

%%% The "real" document content comes below...

\title{%
	Języki Przetwarzania Symbolicznego \\
	\large Projekt : Modyfikacja algorytmu A*
	}
\author{Stawczyk Przemysław}
\date{} % Activate to display a given date or no date (if empty),
% otherwise the current date is printed 


\begin{document}

\maketitle

%\begin{abstract}

%\end{abstract}

\section{Opis zadania}

Wprowadzeniu do programu zmian implementujących program do testowego wykonania algorytmu wg
następującej zasady:
\\\\
Zadane jest ograniczenie liczby kroków -- w postaci dodatkowego argumentu procedury
\textit{search\_A\_star}.
\\\\
W każdym kroku algorytm pobiera z kolejki (uporządkowanej wg oceny heurystycznej stanu), \textit{bez
usuwania}, pierwsze N węzłów -- N jest zadane w wywołaniu procedury \textit{search\_A\_star} (w postaci
dodatkowego argumentu). Lista N węzłów jest przedstawiana użytkownikowi (wyprowadzana
używając \textit{write}), wraz z informacją, który jest to krok wykonania algorytmu. Użytkownik podaje listę
numerów, która określa \textit{kolejność pobierania węzłów z listy węzłów do kolejnych testowych
przebiegów dalszego wykonania algorytmu}. Nazwijmy tę listę numerów listą wyboru. Do dalszego
przebiegu wykonania zostaje pobrany węzeł wskazany pierwszym numerem na liście wyboru.
\\\\
W każdym kroku jest badany i aktualizowany licznik kroków. Jeśli w którymś kroku zostanie
wyczerpany limit liczby kroków, to algorytm nawraca do poprzedniego kroku , aby wybrać na tym
poziomie rekurencji kolejny węzeł z listy węzłów -- \textit{zgodnie z kolejnością podaną na liście wyboru}.
Po wyczerpaniu możliwości wyboru węzła w tym kroku, algorytm pyta użytkownika, czy zwiększyć
limit:
\begin{itemize}
	\item w przypadku odpowiedzi \textit{tak} limit zostaje zwiększony i wykonanie jest kontynuowane (a nie
	rozpoczynane od nowa) z nowym ograniczeniem
	\item w przypadku odpowiedzi \textit{nie} algorytm nawraca do wcześniejszego kroku w celu wyboru innego
	węzła wg listy wyboru
\end{itemize}
(odpowiedź użytkownika jest czytana za pomocą \textit{read},
użytkownik wprowadza sekwencję \textsc{<tekst> <kropka> <ENTER>})
W ramach zadanego ograniczenia liczby kroków są więc badane wszystkie kombinacje możliwych
wyborów węzłów na poszczególnych poziomach rekurencji--na każdym poziomie w kolejności
ustalonej numerami na liście wyboru.

\section{Rozwiązanie}

\subsection{Wynikowy kod}

\verbatiminput{A-star-prolog/A-star-interactive.pl}

\subsection{Przestrzeń stanów}

\verbatiminput{A-star-prolog/roadmap.pl}

\section{Ślad wykonania}

Ślad wykonania dla procedury: \textit{start\_A\_star(a, R, 3, 4).}

\textbf{Wyjście:}
\begin{verbatim}
   search_A_star - Krok nr. : 1
   1 - node(a, nil, nil, 0, 4)
   Podaj indeksy wezlow:
\end{verbatim}
Dostępny jest do wyboru jedynie węzeł startowy - \textit{a}.\\
\textbf{Wejście:}
\begin{verbatim}
   1.
\end{verbatim}
%%%%%%%%%%%%%%%%%%%%%%%%%%%%%%%%%%%%%%%%%
\textbf{Wyjście:}
\begin{verbatim}
   continue - check goal : a
   continue 1 expand : node(a, nil, nil, 0, 4)
   search_A_star - Krok nr. : 2
   1 - node(b, a-b, a, 2, 6)
   2 - node(c, a-c, a, 3, 6)
   Podaj indeksy wezlow:
\end{verbatim}
Próba dopasowania do węzła terminalnego, a następnie rozwinięcie węzła.\\
\textbf{Wejście:}
\begin{verbatim}
   2.
   1.
\end{verbatim}
%%%%%%%%%%%%%%%%%%%%%%%%%%%%%%%%%%%%%%%%%
\textbf{Wyjście:}
\begin{verbatim}
   continue - check goal : c
   continue 2 expand : node(c, a-c, a, 3, 6)
   search_A_star - Krok nr. : 3
   1 - node(b, a-b, a, 2, 6)
   2 - node(d, c-d, c, 5, 6)
   3 - node(e, c-e, c, 6, 10)
   Podaj indeksy wezlow:
\end{verbatim}
Sprawdzany jest pierwszy węzeł z listy wyboru - \textit{node(c, a-c, a, 3, 6)}.\\
\textbf{Wejście:}
\begin{verbatim}
   2.
   3.
   1.
\end{verbatim}
%%%%%%%%%%%%%%%%%%%%%%%%%%%%%%%%%%%%%%%%%
\textbf{Wyjście:}
\begin{verbatim}
   continue - check goal : d
   continue - check goal : e
   continue - check goal : b
   search_A_star - Osiągnięto limit krokow. Zwiekszyc limit? (t/n)
\end{verbatim}
Sprawdzane są wszystkie możliwości na tym poziomie w zadanej kolejności, jako, że osiągnięty został limit kroków. Następnie następuje pytanie do użytkownika - udzielamy odpowiedzi odmownej.\\

\clearpage

\textbf{Wejście:}
\begin{verbatim}
   n.
\end{verbatim}
%%%%%%%%%%%%%%%%%%%%%%%%%%%%%%%%%%%%%%%%%
\textbf{Wyjście:}
\begin{verbatim}
   continue - check goal : b
   continue 2 expand : node(b, a-b, a, 2, 6)
   search_A_star - Krok nr. : 3
   1 - node(c, a-c, a, 3, 6)
   2 - node(g, b-g, b, 6, 7)
   3 - node(f, b-f, b, 5, 12)
   Podaj indeksy wezlow:
\end{verbatim}
Następuje wycofanie się do poziomu niżej i rozwinięcie kolejnego węzła - node(b, a-b, a, 2, 6).\\
\textbf{Wejście:}
\begin{verbatim}
   2.
   3.
   1.
\end{verbatim}
%%%%%%%%%%%%%%%%%%%%%%%%%%%%%%%%%%%%%%%%%
\textbf{Wyjście:}
\begin{verbatim}
   continue - check goal : g
   continue - check goal : f
   continue - check goal : c
   search_A_star - Osiągnięto limit krokow. Zwiekszyc limit? (t/n)
\end{verbatim}
Sprawdzane są wszystkie możliwości na tym poziomie jako, że osiągnięty został limit kroków. Następnie następuje pytanie do użytkownika - udzielamy zgody.\\
\textbf{Wejście:}
\begin{verbatim}
   t.
\end{verbatim}
%%%%%%%%%%%%%%%%%%%%%%%%%%%%%%%%%%%%%%%%%
\textbf{Wyjście:}
\begin{verbatim}
   search_A_star - Krok nr. : 3
   1 - node(c, a-c, a, 3, 6)
   2 - node(g, b-g, b, 6, 7)
   3 - node(f, b-f, b, 5, 12)
   Podaj indeksy wezlow:
\end{verbatim}
Rozwiązywanie jest kontynuowane z nowym limitem.\\
\textbf{Wejście:}
\begin{verbatim}
   3.
   2.
   1.
\end{verbatim}
%%%%%%%%%%%%%%%%%%%%%%%%%%%%%%%%%%%%%%%%%
\textbf{Wyjście:}
\begin{verbatim}
   continue - check goal : f
   continue 3 expand : node(f, b-f, b, 5, 12)
   search_A_star - Krok nr. : 4
   1 - node(c, a-c, a, 3, 6)
   2 - node(g, b-g, b, 6, 7)
   3 - node(h, f-h, f, 9, 12)
   Podaj indeksy wezlow:
\end{verbatim}
Przechodzimy do poziomu 4. Rozwijany jest wybrany węzeł.\\
\textbf{Wejście:}
\begin{verbatim}
   3.
   2.
   1.
\end{verbatim}
%%%%%%%%%%%%%%%%%%%%%%%%%%%%%%%%%%%%%%%%%
\textbf{Wyjście:}
\begin{verbatim}
   continue - check goal : h
   continue - check goal : g
   continue - check goal : c
   search_A_star - Osiągnięto limit krokow. Zwiekszyc limit? (t/n)
\end{verbatim}
Ponownie osiągamy limit głębokości wywołania, sprawdzając wcześniej możliwości z tego poziomu. Odmawiamy zwiększenia limitu.\\
\textbf{Wejście:}
\begin{verbatim}
   n.
\end{verbatim}
%%%%%%%%%%%%%%%%%%%%%%%%%%%%%%%%%%%%%%%%%

\textbf{Wyjście:}
\begin{verbatim}
   continue - check goal : g
   continue 3 expand : node(g, b-g, b, 6, 7)
   search_A_star - Krok nr. : 4
   1 - node(c, a-c, a, 3, 6)
   2 - node(m, g-m, g, 8, 8)
   3 - node(f, b-f, b, 5, 12)
   Podaj indeksy wezlow:
\end{verbatim}
\textbf{Wejście:}
\begin{verbatim}
   2.
   1.
   3.
\end{verbatim}

\textbf{Wyjście:}
\begin{verbatim}
   continue - check goal : m
   continue - reached goal
   R = path_cost([nil/a, (a-b)/b, (b-g)/g, (g-m)/m], 8)
\end{verbatim}
Sprawdzony węzeł prowadzi do stanu docelowego - rozwinięcie ścieżki od węzła startowego do terminalnego.
\end{document}